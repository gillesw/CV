\documentclass[a4paper]{moderncv}

\usepackage[utf8]{inputenc}
\usepackage{eurosym}
\moderncvtheme[blue]{classic}

\firstname{Gilles}
\familyname{Wagner}
\title{Directeur de projets}
\address{31xxxxx}{93xxxxx}
\mobile{+33 6 63 xxxx}
%\phone{+12 (3)456 78 90}
%\fax{+12 (3)456 78 90}
\email{gillxx@xxxx}
%\extrainfo{\weblink{www.ctan.org}}
%\photo[64pt]{jdoe_picture}
%\quote{Any intelligent fool can make things bigger, more complex,
%and more violent. It takes a touch of genius -- and a lot of courage -- to
%move in the opposite direction.}

%37 ans, Marié
%Nationalité Française	Directeur de projets

\begin{document}
\maketitle

\section{Expériences professionnelles}
	%\subsection{Vocational}
		\cventry{Mars 19 -- ...}{Directeur des projets}{Partelya Consulting}{}{}{Direction de projets dans le secteur monétique.\newline{}
		Interface client.\newline{}
		Pilotage de la qualité.\newline{}
		Management d’équipes.}
		\cventry{Août 18 -- Jan. 19}{Senior Manager}{Sopra Steria Consulting}{}{Industry, Transport \& Aéronautique}{Rédaction d’offres de conseil.\newline{}Recrutement.\newline{}Pilotage d’activités de chef de projets, PMO et direction de programmes.\newline{}Management d’équipes.}
		\cventry{Jan. 11 -- Juil. 18}{Manager}{Assystem Group}{}{BU Complex Systems Engineering}{Management d’équipes sur les sujets liés à la gestion de projet.\newline{}
Développement de nouvelles offres.\newline{}
Rédaction des offres sur des sujets liés à la gestion de projet ou au manufacturing engineering.\newline{}
Avant-vente et support aux commerciaux\newline{}
Soutenance client.\newline{}
Prise de commande.\newline{}
Interface client.\newline{}
Recrutement.\newline{}
Formation des équipes.\newline{}
Jusqu’à 20 personnes managées simultanément, jusqu’à 1,5M\euro /an de commande.}
		\cventry{Oct. 15 –- Juil. 18}{Manager de pôle de compétences}{Alstom Transport}{}{}{}
		\cventry{Août 18 -- Jan. 19}{Senior Manager}{Sopra Steria Consulting}{}{}{}
		\cventry{Août 18 -- Jan. 19}{Senior Manager}{Sopra Steria Consulting}{}{}{}
		\cventry{Août 18 -- Jan. 19}{Senior Manager}{Sopra Steria Consulting}{}{}{}
		\cventry{Août 18 -- Jan. 19}{Senior Manager}{Sopra Steria Consulting}{}{}{}
		\cventry{Août 18 -- Jan. 19}{Senior Manager}{Sopra Steria Consulting}{}{}{}
		\cventry{Août 18 -- Jan. 19}{Senior Manager}{Sopra Steria Consulting}{}{}{}
		%\cventry{year--year}{Job title}{Employer}{City}{}{Description}

\end{document}


%Janv. 11 – Juil. 18	Manager – Assystem Group – BU Complex Systems Engineering

%	Manager de pôle de compétences – Alstom Transport
%o	Création et management du service « Common Solutions Pool » : Gestion des exigences, gestion %de la documentation et gestion de la configuration.
%o	Management d’équipes : 18 personnes multisites France et étranger.
%o	Point unique de contact client, gestion de la satisfaction client.
%o	Réalisation des entretiens annuels, fixation des objectifs.
%o	Prestation possible sur tous les sites mondiaux.
%o	Pilotage de l’équipe de support offshore en Inde.
%o	Synergie avec les équipes client.
%o	Centralisation des besoins, interface privilégiée des projets.
%o	Accompagnement du changement.
%o	Gestion de la sous-traitance.
%o	Capture Leader : gestion de la capture des exigences contractuelles et de l’engagement des %métiers sur la conformité en phase d’offres et d’exécution.
%18 personnes, 1,2M€/an pour le service, Conduite du changement sur les programmes, les projets et les sites de production, gain de 2 à 3 mois sur l’engagement de conformité, Mise sous contrôle des métiers.
%Janv 11 – Juil 15	Manager Gestion de projet - Safran Aircraft Engines
%	Chef de projet – Direction Industrielle
%o	Pilotage du bureau d’études de Safran Aircraft Engines, de l’industrialisation, de la %qualité, de la supply chain et des fournisseurs sur 2 rangs.
%o	Pilotage technique de l’appel d’offre vie série.
%Nouvelle définition, retour à l’heure, mise sous contrôle de l’approvisionnement, maitrise des %coûts.
%	PMO – Direction des Moteurs Militaires
%o	Planification du transfert de production en Inde du moteur M88 (contrat MMRCA).
%o	Audit de l’industrialisation.
%o	Support au programme sur la structuration du projet et analyses de risques.
%Amélioration de la maturité de la structuration du projet, réduction de la couverture pour %risques.
%	PMO – Direction des Moteurs Militaires
%o	Planification et pilotage de la réindustrialisation du moteur M53.
%o	Pilotage de fournisseurs concepteurs.
%o	Ordonnancement et gestion des priorités de production.
%Retour à l’heure de l’ensemble du projet, amélioration des délais.
%Oct 08 – Déc10	Consultant – Assystem Group – BU Automobile
%	Conduite de changement LEAN Office – Ingénierie des accessoires
%o	Gestion du changement pour transformer le service en organisation LEAN et instaurer l'état %d'esprit LEAN. 
%o	Identification et résolution des causes de retard.
%Transformation LEAN, mise en place des indicateurs et des rituels de pilotages, retour à %l’heure de l’ensemble des projets suivis.
%	Modélisation de la qualité perçue – Qualité perçue des véhicules
%o	Gestion du projet nouveau référentiel de qualité perçue : conception du modèle avec les %experts, définition du besoin des utilisateurs, gestion des priorités et des urgences, ...
%o	Conception des outils pour aider les responsables qualité perçue des projets véhicules.
%o	Conception des outils d'analyse et des processus.
%Déploiement du modèle à l’ensemble des projets.
%Nov 07 – Sept 08	Modélisation des coûts – Peugeot Citroën Automobiles (Alternance)
%o	Création de modèles, d’outils de chiffrage et de référentiels de coûts de la R&D et les %bureaux d'études à l'usage des analystes chiffreurs et des acheteurs. 
%o	Création de modèles, d'outils de chiffrage et de référentiels de coûts pour les technologies %de décoration intérieure (peinture, tampographie, …)
%Déploiement des modèles à l’ensemble des projets en cours. 15M€ de gains identifiés la première %année.
%Dec 06 – Nov 07	Journaliste indépendant – Presse spécialisée IT
%o	Rédaction d’articles scientifiques et techniques dans la presse spécialisée en ligne et sur %papier.
%o	Rédaction de tests de matériel et de benchmarks en ligne et sur papier.
%Plus de 100000 lecteurs sur Internet, plus de 80000 lecteurs sur papier.
%Formation
%2007 – 2008	Mastère spécialisé « Management des projets technologiques » – Brest Business %School
%2005 – 2006	Master de recherche « Sciences des matériaux » – Université de Caen
%2002 – 2006	Diplôme d’ingénieur « Matériaux et chimie fine » – ENSICAEN
%Langues
%Anglais	Courant
%Allemand	Intermédiaire
%Divers
%	Habilitations H0 B0 et M0 (1999)

%Judo (Ceinture noire, 10 ans de compétition), Moto (Route et circuit).
