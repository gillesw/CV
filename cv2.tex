\documentclass[a4paper,10pt]{moderncv}

\usepackage[utf8]{inputenc}
\usepackage{eurosym}
\moderncvtheme[blue]{classic}

%\newcommand{\subcventry}{•}
\usepackage[left=2cm,right=1.9cm,top=2cm,bottom=2cm]{geometry}
\setlength{\hintscolumnwidth}{2.8cm}
%\newcommand\Mycventry[6]{\cventry{#1}{#2}{#3}{#4}{#5}{\Huge#6}}  % this works

\firstname{Gilles}
\familyname{Wagner}
\title{Directeur des projets}
\address{31xxxxx}{93xxxxx}
\mobile{+33 6 63 xxxx}
%\phone{+12 (3)456 78 90}
%\fax{+12 (3)456 78 90}
\email{gillxx@xxxx}
%\extrainfo{\weblink{www.ctan.org}}
%\photo[64pt]{jdoe_picture}
%\quote{Any intelligent fool can make things bigger, more complex,
%and more violent. It takes a touch of genius -- and a lot of courage -- to
%move in the opposite direction.}

%37 ans, Marié
%Nationalité Française

\begin{document}
\maketitle

\section{Expériences professionnelles}
	%\subsection{Vocational}
		\cventry{Mars 19 -- \ldots}{Directeur des projets}{Partelya Consulting}{}{}{
			Direction de projets dans le secteur monétique.\\
			Interface client.\\
			Pilotage de la qualité.\\
			Management d’équipes.}
		\cventry{Août 18 -- Jan. 19}{Senior Manager}{Sopra Steria Consulting}{Industry, Transport \& Aéronautique}{}{
			Rédaction d’offres de conseil.\\
			Recrutement.\\
			Pilotage d’activités de chef de projets, PMO et direction de programmes.\\
			Management d’équipes.}
		\cventry{Jan. 11 -- Juil. 18}{Manager}{Assystem Group}{BU Complex Systems Engineering}{}{
			Management d’équipes sur les sujets liés à la gestion de projet.\\
			Développement de nouvelles offres.\\
			Rédaction des offres sur des sujets liés à la gestion de projet ou au manufacturing 				engineering.\\
			Avant-vente et support aux commerciaux\\
			Soutenance client.\\
			Prise de commande.\\
			Interface client.\\
			Recrutement.\\
			Formation des équipes.\\
			Jusqu’à 20 personnes managées simultanément, jusqu’à 1,5M\euro /an de commande.}
		\cventry{Oct. 15 –- Juil. 18}{Manager de pôle de compétences}{Alstom Transport}{}{}{
			Création et management du service « Common Solutions Pool » : Gestion des exigences, 			gestion de la documentation et gestion de la configuration.\\
			Management d’équipes : 18 personnes multisites France et étranger.\\
			Point unique de contact client, gestion de la satisfaction client.\\
			Réalisation des entretiens annuels, fixation des objectifs.\\
			Prestation possible sur tous les sites mondiaux.\\
			Pilotage de l’équipe de support offshore en Inde.\\
			Synergie avec les équipes client.\\
			Centralisation des besoins, interface privilégiée des projets.\\
			Accompagnement du changement.\\
			Gestion de la sous-traitance.\\
			Capture Leader : gestion de la capture des exigences contractuelles et de 							l’engagement des métiers sur la conformité en phase d’offres et d’exécution.\\
			18 personnes, 1,2M\euro /an pour le service, Conduite du changement sur les 						programmes, les projets et les sites de production, gain de 2 à 3 mois sur 							l’engagement de conformité, Mise sous contrôle des métiers.}
		\cventry{Janv. 11 –- Juil. 15}{Manager Gestion de projet}{Safran Aircraft Engines}{}{}{}
		\cventry{}{Chef de projet}{Direction Industrielle}{}{}{
			Pilotage du bureau d’études de Safran Aircraft Engines, de l’industrialisation, de
			la qualité, de la supply chain et des fournisseurs sur 2 rangs.//
			Pilotage technique de l’appel d’offre vie série.//
			Nouvelle définition, retour à l’heure, mise sous contrôle de l’approvisionnement,
			maitrise des coûts.}
		\cventry{}{PMO}{Direction des Moteurs Militaires}{}{}{
			Planification du transfert de production en Inde du moteur M88 (contrat MMRCA).\\
			Audit de l’industrialisation.\\
			Support au programme sur la structuration du projet et analyses de risques.\\
			Amélioration de la maturité de la structuration du projet, réduction de la
			couverture pour risques.}
		\cventry{}{PMO}{Direction des Moteurs Militaires}{}{}{
			Planification et pilotage de la réindustrialisation du moteur M53.\\
			Pilotage de fournisseurs concepteurs.\\
			Ordonnancement et gestion des priorités de production.\\
			Retour à l’heure de l’ensemble du projet, amélioration des délais.}
		\cventry{Oct 08 –- Déc10}{Consultant}{Assystem Group}{BU Automobile}{}{}{}
		\cventry{}{Conduite de changement LEAN Office}{Peugeot Citroën Automobiles}{Ingénierie des accessoires}{}{
			Gestion du changement pour transformer le service en organisation LEAN et instaurer
			l'état d'esprit LEAN.\\
			Identification et résolution des causes de retard.\\
			Transformation LEAN, mise en place des indicateurs et des rituels de pilotages,
			retour à l’heure de l’ensemble des projets suivis.}
		\cventry{}{Modélisation de la qualité perçue}{Peugeot Citroën Automobiles}{Qualité perçue des véhicules}{}{
			Gestion du projet nouveau référentiel de qualité perçue : conception du modèle avec
			les experts, définition du besoin des utilisateurs, gestion des priorités et des
			urgences, ...\\
			Conception des outils pour aider les responsables qualité perçue des projets
			véhicules.\\
			Conception des outils d'analyse et des processus.\\
			Déploiement du modèle à l’ensemble des projets.}
		\cventry{Nov 07 –- Sept 08}{Modélisation des coûts}{Peugeot Citroën Automobiles}{}{}{
			Création de modèles, d’outils de chiffrage et de référentiels de coûts de la R\&D et 			les bureaux d'études à l'usage des analystes chiffreurs et des acheteurs.\\
			Création de modèles, d'outils de chiffrage et de référentiels de coûts pour les
			technologies de décoration intérieure (peinture, tampographie, \ldots )\\
			Déploiement des modèles à l’ensemble des projets en cours. 15M\euro de gains
			identifiés la première année.}
		\cventry{Dec. 06 –- Nov. 07}{Journaliste indépendant}{Presse spécialisée IT}{}{}{
			Rédaction d’articles scientifiques et techniques dans la presse spécialisée en ligne
			et sur papier.\\
			Rédaction de tests de matériel et de benchmarks en ligne et sur papier.\\
			Plus de 100000 lecteurs sur Internet, plus de 80000 lecteurs sur papier.}
		
		%\cventry{year--year}{Job title}{Employer}{City}{}{Description}
\section{Formation}
	\cventry{2007 -- 2008}{Mastère spécialisé « Management des projets technologiques »}{
		Brest Business School}{}{}{}
	\cventry{2005 -- 2006}{Master de recherche « Sciences des matériaux »}{
		Université de Caen}{}{}{}
	\cventry{2002 -- 2006}{Diplôme d’ingénieur « Matériaux et chimie fine »}{ENSICAEN}{}{}{}
\section{Langues}
	\cvitem{Anglais}{Courant}
	\cvitem{Allemand}{Intermédiaire}
\section{Divers}
	%\cvitem{}{Habilitations H0 B0 et M0 (1999)}
	%\\
	\cvitem{}{Judo (Ceinture noire, 10 ans de compétition),	Moto (Route et circuit).}
\end{document}


